\documentclass{beamer}

\usepackage[utf8]{inputenc}

\title{Présentation finale projet 2}
\author{Maxime Darrin  \and Edwige Cyffers }
\institute{}
\date{}


\usetheme{CambridgeUS}
\usecolortheme{spruce}

\begin{document}

\maketitle

\begin{frame}
\tableofcontents
\end{frame}


\section{Le résultat de notre travail}

\begin{frame}{Notre fouine}

  \begin{itemize}
  \item L'interpréteur marche bien \pause
  \item Les traductions aux trois quarts \pause
  \item La machine marche bien, et sinon les erreurs sont jolies \pause
  \item L'inférence de types marche sur les cas décents.\pause

  \end{itemize}
  
\end{frame}

\section{Le travail en groupe}

\begin{frame}{De l'importance du travail en équipe}
	Il parait qu'on ne pas tout créer tout seul. 
	
	Surtout quand on n'est pas Sébastien.
\end{frame}

\begin{frame}{Les points délicats}
	\begin{itemize}
		\item Les méthodes de travail de l'autre \pause
		\item La médiocrité de l'autre \pause
		\item La fainéantise de l'autre \pause
		\item La rapidité de l'autre\pause
		\item Le code de l'autre \pause
	\end{itemize}
	
\end{frame}

\begin{frame}{Quelques suggestions}
	\begin{itemize}
		\item Savoir ce qu'on sait faire, ce qu'on peut apprendre, et ce qui est loin de nous \pause
		\item Faire une répartition des tâches avec des échéances précises à respecter\pause
		\item Discuter régulièrement de l'avancement et des choix faits\pause
		\item Partager les tâches pénibles\pause
		\item Ne pas en vouloir à son binôme lorsqu'il ne respecte aucun des points précédents \pause
	\end{itemize}

\end{frame}

\end{document}



