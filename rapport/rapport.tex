\documentclass{article}

\usepackage[utf8]{inputenc}


\usepackage[french]{babel}


\title{Rapport projet 2}
\author{Maxime Darrin  \and Edwige Cyffers }
\date{}


\begin{document}

\maketitle

\section{Présentation}

	Au commencement était un petit tableur tout innocent. Un souvenir ému du projet d'asr1, et la conviction que l'on pouvait reproduire un aussi belle expérience. Depuis, bien sûr, les épreuves se sont enchaînées, les difficultés ont parfois été un peu plus ardues que prévues, mais nous vous rendons tout de même un rendu avancé qui fonctionne correctement, malgré une catégorisation initiale en intermédiaires.
	
	Par rapport aux exigences du rendu avancé, nous avons traité les couples comme étant des n-uplets dans toutes les parties du projet. Fouine comporte également le matching, les constructeurs (sous réserve de toujours mettre des parenthèses) et les listes. enfin, on peut noter une tentative d'inférence de type.


	Nous exposons à la partie~\ref{s:orga} comment notre programme est structuré.


\section{Organisation du code}
\label{s:orga}

Le code est structuré de la manière suivante~:
\begin{itemize}
\item l'ensemble du code est dans scr, nous n'avons pas séparés les différentes parties du projet.
\item les tests sont par contre séparés en différents dossiers selon leurs buts, et possèdent des scripts associés. \cite{Landin:1966:NPL:365230.365257}

\end{itemize}

\section{Critique des performances}

	Je te laisse faire ton TIPE.
	
\section{Conclusion}
	Je te laisse ce privilège aussi.

\bibliographystyle{plain}
\bibliography{biblio}

\end{document}
